\documentclass[UTF8]{ctexbeamer}
\usepackage{hyperref}
\usetheme{Berlin}

\title{《信号与系统》习题讨论:7.3}
\author{高浚哲}
\begin{document}
  
  \begin{frame}
    \maketitle
  \end{frame}

  \begin{frame}{题目}
    系统函数$H(s)$的零点在$1\pm j$,极点在$-1\pm j$,且$H(0)=3$,写出其$H(s)$的表示式。
  \end{frame}

  \begin{frame}{解题思路}
    集总参数LTI系统的系统函数是复变量$s$或$z$的有理分式,即
    $$
      H(\cdot) = \frac{B(\cdot)}{A(\cdot)}
    $$
    对于本题的连续系统,则有:
    $$
      H(s)=\frac{b_m s^m+b_{m-1} s^{m-1}+\dots+b_1 s + b_0}{s^n+a_{n-1}s^{n-1}+\dots+a_1s+a_0}
    $$
    由零点和极点定义可知:零点即为方程$A(s)=0$的解,极点为$B(s)=0$的解;则系统函数可进一步写作:
    $$
      H(s)=\gamma \frac{(s-\beta_1)(s-\beta_2)\dots(s-\beta_m)}{(s-\alpha_1)(s-\alpha_2)\dots(s-\alpha_n)}
    $$
  \end{frame}

  \begin{frame}{解题思路}
    在本题中,由于已知系统函数的零点和极点,系统函数可写作:
    $$
      H(s)=\gamma \frac{(s+1)^2+1}{(s-1)^2+1}
    $$
    题中给出条件:$H(0)=3$,将其带入,则有:
    $$
      H(0)=\gamma \frac{(0 + 1)^2+1}{(0-1)^2+1}=3
    $$
    解之得:$\gamma = 3$,即:
    $$
      H(s)=3 \frac{(s+1)^2+1}{(s-1)^2+1}
    $$
  \end{frame}
\end{document}
